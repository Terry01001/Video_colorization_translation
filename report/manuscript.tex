\documentclass[sigconf]{acmart}

\usepackage{fontspec}
\setmainfont{Times New Roman}
%\usepackage{CJKutf8}
\usepackage{caption}
\usepackage[ruled,linesnumbered]{algorithm2e}
\usepackage{color}
\usepackage{indentfirst}
\usepackage{natbib}
\usepackage{graphicx}
\usepackage{float}
\usepackage{enumitem}
\usepackage{titlesec}
\usepackage{tabularx}
\usepackage{amsmath}
\usepackage{multirow}
\usepackage{colortbl}
\usepackage{amsthm}
\usepackage{xeCJK}
\usepackage{booktabs}
\usepackage{diagbox}
\usepackage{subcaption} % 或者使用 subfigure 宏包
\setCJKmainfont{DFKai-SB} % 在 Windows 上設置標楷體

\newtheorem{definition}{Definition}


%\let\listofalgorithms\relax

%\usepackage{algorithm}
%\usepackage{algorithmicx}
\usepackage{algpseudocode}


\titleformat{\section}[hang]{\bfseries\Large}{\thesection}{1em}{}




\setlength{\parindent}{2em}
\newcommand{\xfig}[1]{圖~\ref{#1}}
\newcommand{\xfigs}[2]{圖~\ref{#1} and~\ref{#2}}
\newcommand{\xq}[1]{\textcolor{red}{#1}}

\newcommand{\xmold}[1]{\textcolor{red}{#1}} % original text
\newcommand{\xmnew}[1]{\textcolor{blue}{#1}} % replacement text
\newcommand{\xch}[2]{\xmold{\sout{#1}}\xmnew{#2}}


\newcommand{\xmesni}{He-AdaBoost with varying $\gamma$ values in RBFSVM}
\newcommand{\xmesnis}{He-AdaBoost$\mathbf{_{v\gamma}}$}
\newcommand{\xmesnia}{{\xmesni} (\xmesnis)}

\newcommand{\xmesniabstract}{He-AdaBoost with varying $\gamma$ value in SVM with radial basis function (RBFSVM)}

\begin{document}



\renewcommand\thesection{\arabic{section}}

%\begin{CJK}{UTF8}{bkai}
\title{人工智慧導論 第二組期末報告}
\author{張碩文}
\affiliation{%
  \institution{B093040007}
  \city{Kaohsiung}
  \country{Taiwan}
}
\author{陳昱逢}
\affiliation{%
  \institution{B092040016}
  \city{Kaohsiung}
  \country{Taiwan}
}

\author{黃子耘}
\affiliation{%
  \institution{B094020037}
  \city{Kaohsiung}
  \country{Taiwan}
}
\author{李妙華}
\affiliation{%
  \institution{B093040072}
  \city{Kaohsiung}
  \country{Taiwan}
}
\author{陳義竑}
\affiliation{%
  \institution{B123040018}
  \city{Kaohsiung}
  \country{Taiwan}
}



\begin{abstract}
  % Problem  
	我們首先使用 perlin noise 隨機生成模擬自然環境,地圖上各點會依據不同環境,而有不同距離權重。本實驗的目標是利用蟻群演算法,找到螞蟻從圖中左上角到右下角的最佳化路徑,並且嘗試各種超參數去分析此地圖下哪種情況下最為合適。

\end{abstract}




\maketitle


  
\section{簡介}
\label{sec:introduction}


	想模擬一自然環境,並利用超參數的不同以及不同距離權重還有費洛蒙的保存率,來觀察實驗結果會有何不同的結果,並利用pygame來視覺化這些實驗的操作跟結果。
 
\iffalse
\begin{figure}[htb]
  \vspace{-\baselineskip}  
  \centering  
  %\begin{center}
    \resizebox{0.35\textwidth}{!}{\includegraphics{fig/VANET_communication.pdf}}
    \caption{Communication types in C-ITS}
    \label{fig:vanet_communication}
  %\end{center}
  \vspace{-\baselineskip}
\end{figure}	 
\fi
	


\section{相關研究}
\label{sec:related_work}


\subsection{地圖的生成}

	如用隨機生成之地圖,地圖混亂度會很高,很像雜訊很多的圖片,不符合正常自然環境會存在的地圖;但使用perlin noise之後讓地圖更加有連續性,如此才有一片的草地、泥土、沙地,在 perlin noise 套件中可以設定octave參數,參數較大時,可以讓地圖中草地、泥土、沙地切割的區塊更細緻跟均勻散佈在地圖上。

\begin{figure}[htb]
  \centering
  \begin{subfigure}{0.2\textwidth}
    \centering
    \includegraphics[width=\textwidth]{fig/map_noise.png} % 插入第一張圖片,替換 "image1" 為實際的文件名
    \caption{Before using perlin noise}
    \label{fig:sub1}
  \end{subfigure}
  \hfill
  \begin{subfigure}{0.2\textwidth}
    \centering
    \includegraphics[width=\textwidth]{fig/map.png} % 插入第二張圖片,替換 "image2" 為實際的文件名
    \caption{After using perlin noise}
    \label{fig:sub2}
  \end{subfigure}
\end{figure}



	
\subsection{蟻群演算法規則及定義}

	有群螞蟻要從起點走到終點,在走路的過程中將會留下費洛蒙,而在越短距離內走完,則費洛蒙則會最濃,相反的越長距離走完則費洛蒙則會隨著不同區塊會有不同的蒸發率,而隨之變淡。以下有兩個公式來計算如何讓螞蟻判別該往何處移動。


	公式 (\ref{eqn:update_pheremon}) 的上式在描述螞蟻的費洛蒙保存率更新,定義一常數為 $\rho$ 為費洛蒙揮發常數,則 $1- \rho$ 則表保存。而下式即為第 $k$ 只螞蟻的費洛蒙更新量。

\begin{equation}
  \label{eqn:update_pheremon}
\left\{\begin{array}{ll}
                 \tau_{ij}(t + 1) &= (1 - \rho)\tau_{ij}(t) + \Delta\tau_{ij} \\
                 \Delta\tau_{ij} &= \sum_{k=1}^{n} \delta\tau_{ij}^{k}
                \end{array} \right.
\end{equation}




公式 (\ref{eqn:path_choose}) 表示路徑選擇公式。	其中上式 $P^k_{ij}(t)$ 表第 $k$ 只螞蟻從起點走向終點的機率,$s$ 是第 $k$ 只螞蟻能抵達的所有點。$\tau_{ij}(t)$ 表 $t$ 是時刻,從起點到終點的費洛蒙濃度,$\eta_{is}(t)$ 表 $t$ 是時刻,從起點到終點的距離的倒數。$\alpha , \beta $ 則是我們可以定義的參數,也是本次實驗的重點,若想著重於費洛蒙的影響則可以調整 $\alpha$ 的大小,反之,若想看距離的影響則可以調整 $\beta$ 的大小。



\begin{equation}
  \label{eqn:path_choose}
P^k_{ij}(t) = \left\{\begin{array}{ll}
				\frac{\tau_{ij}(t)^\alpha \cdot \eta_{ij}(t)^\beta}{\sum_{s \in \text{allow}_k}  \tau_{is}(t)^\alpha \cdot \eta_{is}(t)^\beta,} & \text{if } s \in \text{allow}_k \\ 
                0, & \text{if } s \notin \text{allow}_k.
                \end{array} \right.
\end{equation}

\subsection{螞蟻的規則}

\begin{enumerate}[label=\textbf{\arabic*.}]
	\item 螞蟻可在地圖上可以走上、下、左、右、四個斜對角共八個方向。
	\item 不能重複走相同的點。
	\item 必須要成功地從起點走到終點才能留下自身的費洛蒙。
	\item 若在草地、泥土、沙地的費洛蒙保存率可依據自身設定而有所不同。
	\item 點跟點之間的距離要依據這兩個點本身的環境去計算 ( Ex : 從草地走到泥土...)。
	\item 當地圖某個位置的費洛蒙濃度低於 10 的 -5 次方時,會將該位置的濃度依據本身的環境設定初始值。
\end{enumerate}

\subsection{螞蟻行走距離權重及費洛蒙重置}

我們的地圖有三個不同的區塊分別是草地、泥土、沙子,每一塊有著不同的距離權重,分別為1、3 和 5,表格 \ref{table:path_weight} 呈現路徑的權重。

\begin{table}[htb]
	\centering	
	\caption{路徑的距離權重}
	\vspace{-\baselineskip}
	\label{table:path_weight}
	\begin{tabularx}{0.5\textwidth}{|c|*{3}{>{\centering\arraybackslash}X|}}
		\hline
		\diagbox{起點}{終點} & \textbf{草地} & \textbf{泥土} & \textbf{沙子} \\ \hline
		\textbf{草地}     & 2       & 4     & 6         \\ \hline
		\textbf{泥土}     & 4       & 6     & 8         \\ \hline
		\textbf{沙子}     & 6       & 8     & 10        \\ \hline
	\end{tabularx}
\end{table}

當費洛蒙濃度低於 10 的 -5 次方,就要 reset 費洛蒙,各個環境的預設值如下:
\begin{enumerate}[label=\textbf{\arabic*.}]
	\item 草地費洛蒙: 1 / 草地距離權重 = $\frac{1}{1}$
	\item 泥土費洛蒙: 1 / 泥土距離權重 = $\frac{1}{3}$
	\item 沙地費洛蒙: 1 / 沙地距離權重 = $\frac{1}{5}$
\end{enumerate}

重設的目的是為了讓螞蟻跳出當前的 local optimal


\section{Demo}
\subsection{操作界面}


\begin{figure}[htb]
  \vspace{-\baselineskip}
  \centering  
  %\begin{center}
    \resizebox{0.5\textwidth}{!}{\includegraphics{fig/UI.png}}
    \caption{User Interface}
    \label{fig:1}
  %\end{center}
  \vspace{-\baselineskip}
\end{figure}



\subsection{功能表}


\vspace{-2mm}
\begin{table}[H]
	\centering	
	\normalsize
    \newcommand{\z}{\phantom{0}}
    \caption{功能表}
    \vspace{-\baselineskip}
    \resizebox{0.5\textwidth}{!}{
		\begin{tabular}[c]{|l|l|}
			\hline
			Start/Pause & 開始/暫停 \\
			\hline
			Clear & 清空畫面\\
			\hline
			grass & 設定費洛蒙在草地的蒸發率 $\pm 0.1$ \\
			\hline
			soil & 設定費洛蒙在泥土的蒸發率 $\pm 0.1$\\
			\hline
			sand & 設定費洛蒙在沙子的蒸發率 $\pm 0.1$ \\
			\hline
			alpha & 螞蟻重視費洛蒙的程度 $\pm 1.0$ \\
			\hline
			beta & 螞蟻重視距離的程度 $\pm 1.0$ \\
			\hline
			stop & 設定當跑到第幾個generation的時要停止程式 $\pm 50$ \\
			\hline
			space (空白鍵) & 切換費洛蒙 (紅點) 是否要顯示 \\
			\hline
		\end{tabular}
	}
	\label{table:feature}
   \vspace{-\baselineskip}
\end{table}




\section{螞蟻參數實驗}
\label{sec:proposed}

\subsection{所使用的背景地圖}

定義一個 40 * 40 二維網格世界,並利用 Perlin Noise 且 Octave 成 8,且測試三種隨機種子地圖(seed=100、200、300)。


\subsection{參數設定}

一個實驗的 generation 會有 200 隻螞蟻,Q 設定 100,並在固定的十個種子碼下,跑完 750 個 generation 即停止搜尋。
 本實驗嘗試了三組保存率,組合一(距離權重越低則費洛蒙保存率越高)、組合二(不論距離權重為何,費洛蒙保存率皆相同)、組合三(距離權重越高則費洛蒙保存率越高);


本實驗中 $\alpha + \beta = 6$,總共測試了五組($\alpha , \beta$) = (1, 5)、(2, 4)、(3, 3)、(4, 2)、(5, 1),所以($\alpha , \beta$) 和費洛蒙保存率的組合共有 15 組參數可以測試。


\subsection{實驗結果呈現}

\begin{table}[htb]
	\centering
	\small
	\caption{MAP\_100}
	\vspace{-\baselineskip}
	\label{table:path_weight}
	\begin{tabularx}{0.5\textwidth}{|c|*{5}{>{\centering\arraybackslash}X|}}
		\hline
		\diagbox{組合}{($\alpha$,$\beta$)}   & \textbf{(1,5)}       & \textbf{(2,4)}        & \textbf{(3,3)} & \textbf{(4,2)} & \textbf{(5,1)}   \\ \hline
		\textbf{組合一}                      & $684 \pm 29$   & $608 \pm 18$    & $469 \pm 28$   & $486 \pm 29$  & $558 \pm 34$         \\ 
		\textbf{組合二}                      & $532 \pm 28$   & $350 \pm 25$    & \textbf{229} $\boldsymbol{\pm}$  \textbf{28}   & $537 \pm 50$  & $583 \pm 48$         \\ 
		\textbf{組合三}                      & $435 \pm 24$   & $326 \pm 13$   & $262 \pm 20$   & $536 \pm 30$  & $584 \pm 35$          \\  \hline
	\end{tabularx}
\end{table}

\begin{figure}[htb]
  \vspace{-\baselineskip}  
  \centering  
  %\begin{center}
    \resizebox{0.5\textwidth}{!}{\includegraphics{fig/MAP100.png}}
    \caption{MAP\_100}
    \label{fig:1}
  %\end{center}
  \vspace{-\baselineskip}
\end{figure}

\vspace{-2mm}
\begin{table}[htb]
	\centering
	\small
	\caption{MAP\_200}
	\vspace{-\baselineskip}
	\label{table:path_weight}
	\begin{tabularx}{0.5\textwidth}{|c|*{5}{>{\centering\arraybackslash}X|}}
		\hline
		\diagbox{組合}{($\alpha$,$\beta$)}   & \textbf{(1,5)}       & \textbf{(2,4)}        & \textbf{(3,3)} & \textbf{(4,2)} & \textbf{(5,1)}   \\ \hline
		\textbf{組合一}                      & $556 \pm 54$   & $558 \pm 34$    & $451 \pm 76$   & $450 \pm 24$  & $474 \pm 38$         \\ 
		\textbf{組合二}                      & $465 \pm 23$   & $319 \pm 30$    & \textbf{233} $\boldsymbol{\pm}$  \textbf{66}   & $514 \pm 51$  & $517 \pm 35$         \\ 
		\textbf{組合三}                      & $430 \pm 30$   & $343 \pm 16$   & $461 \pm 72$   & $512 \pm 45$   & $520 \pm 38$          \\  \hline
	\end{tabularx}
\end{table}

\begin{figure}[htb]
  \vspace{-\baselineskip}  
  \centering  
  %\begin{center}
    \resizebox{0.5\textwidth}{!}{\includegraphics{fig/MAP200.png}}
    \caption{MAP\_200}
    \label{fig:2}
  %\end{center}
  \vspace{-\baselineskip}
\end{figure}

\begin{table}[H]
	\centering
	\small
	\caption{MAP\_300}
	\vspace{-\baselineskip}
	\label{table:path_weight}
	\begin{tabularx}{0.5\textwidth}{|c|*{5}{>{\centering\arraybackslash}X|}}
		\hline
		\diagbox{組合}{($\alpha$,$\beta$)}   & \textbf{(1,5)}       & \textbf{(2,4)}        & \textbf{(3,3)}                           & \textbf{(4,2)}   & \textbf{(5,1)}      \\  \hline
		\textbf{組合一}                      & $469 \pm 45$   & $445 \pm 46$                & $291 \pm 52$                             & $395 \pm 22$    & $432 \pm 20$         \\ 
		\textbf{組合二}                      & $391 \pm 29$   & $270 \pm 13$                & \textbf{160} $\boldsymbol{\pm}$  \textbf{9}       & $406 \pm 38$    & $464 \pm 22$         \\ 
		\textbf{組合三}                      & $327 \pm 26$   & $256 \pm 9$                 & $171 \pm 5$                              & $250 \pm 98$    & $454 \pm 42$          \\  \hline
	\end{tabularx}
\end{table}

\begin{figure}[H]
  \vspace{-\baselineskip}  
  \centering  
  %\begin{center}
    \resizebox{0.5\textwidth}{!}{\includegraphics{fig/MAP300.png}}
    \caption{MAP\_300}
    \label{fig:3}
  %\end{center}
  \vspace{-\baselineskip}
\end{figure}

	

\section{實驗結果分析}

根據上方三張不同種子碼的地圖中皆可大致觀察出一個趨勢,不論費洛蒙保留率的組合為何,當($\alpha$,$\beta$)=(3,3) 的時候,較容易找到更佳的解。並且在3個隨機種子地圖來看,皆是環境費洛蒙保留率皆為 0.6 時,再加上($\alpha$,$\beta$)=(3,3)時,這樣的參數組合會是15組中最好的結果。




\section{結論和未來展望}
\label{sec:simulation}

在本實驗的地圖中,雖然各個環境下各自的環境權重,我們原本預計賀爾蒙的影響會比距離的影響更為重要。但在我們的實驗中,更好的解是各個環境(泥土、草地、沙子)的費洛蒙保留率要一樣,且($\alpha$,$\beta$)的組合反而是費洛蒙跟距離重要性一樣為最好。
 希望在未來可以嘗試不同Noise 方式生成地圖,螞蟻選擇路徑的方式,新增螞蟻的天敵,讓實驗更加貼近真實環境。



\bibliographystyle{IEEEtran} 
\bibliography{reference}

\nocite{ACO}
\nocite{ANT}
\nocite{EMOJI}

\end{document}